% -*- coding: latin-1 -*-
%
% Copyright 2005 Fundación Via Libre
%
% This file is part of PAPO.
%
% PAPO is free software; you can redistribute it and/or modify it under the
% terms of the GNU General Public License as published by the Free Software
% Foundation; either version 2 of the License, or (at your option) any later
% version.
%
% PAPO is distributed in the hope that it will be useful, but WITHOUT ANY
% WARRANTY; without even the implied warranty of MERCHANTABILITY or FITNESS
% FOR A PARTICULAR PURPOSE.  See the GNU General Public License for more
% details.
%
% You should have received a copy of the GNU General Public License along with
% PAPO; if not, write to the Free Software Foundation, Inc., 59 Temple Place,
% Suite 330, Boston, MA 02111-1307 USA
% 
\documentclass[a4paper]{howto}

\setlength\textwidth{6.5in}
\setlength\oddsidemargin{0in}
\setlength\evensidemargin{0in}

%\usepackage[spanish]{babel}
\usepackage[latin1]{inputenc}

\newcommand{\Cimarron}[0]{Cimarr�n}

\title{\Cimarron Tutorial}
\author{Marcos Dione and Federico Heinz\\Fundaci�n V�a Libre}
\authoraddress{\email{mdione@vialibre.org.ar}\\\email{fheinz@vialibre.org.ar}}
%\release{0.00}

\begin{document}

\maketitle

\ifhtml
\chapter*{Front Matter}
\fi

\begin{abstract}
 \noindent
 This document is intended as a primer for Python programmers who want
 to get an overview on how to write programs based on \Cimarron.

 \Cimarron\ is a simplified framework designed to help write
 cross-platform database-centric applications, such as accounting
 systems. It provides an abstract widget set that can be presented to
 the user through the native user interface on the system she is
 running, and is heavily based on the Model-View-Controller and the
 Delegation patterns.

 After reading this document, you should have a general understanding of
 \Cimarron's way of doing things.
\end{abstract}

\tableofcontents

\section{Before you even begin}

Python will have to be able to find \Cimarron{} in order for the
examples in this tutorial to work. Unless \Cimarron{} is installed in
a directory within your \envvar{PYTHONPATH}, you may have to run your
instance of Python with a \envvar{PYTHONPATH} that works. Assuming
your copy of \Cimarron{} is in \file{\textasciitilde/src/cimarron},
the following command sequence should do the trick:

\begin{verbatim}
~$ cd src/cimarron
~/src/cimarron$ export PYTHONPATH=".:./examples/person:$PYTHONPATH"
~/src/cimarron$ python
\end{verbatim}

\section{First steps}

Here is the Python code to produce the GUI equivalent of the ``Hello
Word'' program using \Cimarron. Just paste the code into an instance of
Python to get a single window with a single button in it. The window's
title is ``Hello World Demo'', and the button reads ``Hello''.

\verbatiminput{helloworld.py}

As you can see in the code, we build the window from the outside in,
specifying each new widget's parent as we go. We start declaring
\class{MainWindowController} as a subclass of
\class{WindowController}. As such, every instance will already have a
window associated with it through the \member{win} attribute. The
constructor for \class{MainWindowController} sets the window's title and
size. Then it creates a \class{Button} instance, while specifying the
window as its parent and the string ``Hello'' as its label.

By convention, you can initialize all the attributes you want on an
instance of a \Cimarron{} object by passing the appropriate keyword
parameters when creating it. For this to work, it is important that
whenever you subclass a \Cimarron{} class, you include the parameter
\var{**kw} in the class' constructor, and that you pass that parameter
along to your superlclass' constructor, as shown in the declaration of
\class{MainWindowController}'s \method{__init__()}. It is worth noting
that the constructor does no checking on the attributes you set this
way, so specifying an undefined attribute will lead to its creation:

\begin{verbatim}
>>> from fvl import cimarron
>>> a = Button(foo=42)
>>> a.foo
42
\end{verbatim}

The program doesn't do anything beyond showing these elements. If the
user interface you are using allows for it, you can use the standard
mechanisms to resize the window, close it, etc. You can even activate
the button, but nothing special will happen. It is interesting to note
that whether you actually see a graphical, textual or web interface
depends on the environment under which you are running the
program. \Cimarron{} provides a set of abstract widgets used to display
and input data. When you run the program, the system decides which user
interface is the most appropriate for your environment, and displays it
accordingly\footnote{At the moment, this is true only in theory:
\Cimarron{} is designed to work that way, but only the GTK2 engine and
available. There is also an experimental and incomplete Qt engine, and
writing new ones is a fairly straightforward task.  Of course, we
wellcome contributions for additional engines, including engines to
enable \Cimarron{} programs to run on proprietary platforms.}.

The rest of the code is the main application. Every \Cimarron{} program
needs an instance of \class{Application} (or of a custom subclass of it)
to take care of interaction with the user. After storing the instance of
\class{Application} in \var{app}, the program creates an instance of
\class{MainWindowController} and puts it on screen with a call to its
\method{show()} method. It then calls \class{Application}'s
\method{run()} method to handle user events. This method exits when the
user closes the last (and in this case only) window.

As an exercise, we suggest that you attempt to modify the code above in
such a way that \class{MainWindowController}'s constructor accepts a
parameter \var{title} that allows you to specify the window's title on
creation, with a default value of ``Main Window''. Then modify the main
program to display two (or more) windows with different titles.

\section{Actions}

To get the program to do something more interesting than just displaying
widgets, we can use the \class{Button}'s action. All \Cimarron{}
controls can have an action associated with them. The action is a
callable object that will be invoked by the control whenever it is
triggered (each kind of control defines what it takes to trigger
it). The called function must accept one argument, which will contain
the control that got triggered. This way, the same action can be shared
by different controls.

The following code extends the first example to actually do something
when the button is pressed: it prints the message ``Button pressed!'' in
the console. Not very useful, but enough for the purpose of
illustration.

\verbatiminput{simpleaction.py}

Note that although the \method{doSomething()} method ignores the
\var{sender} parameter, it could query the \class{Button} it contains
for information, or otherwise alter it. For example, this program shows
how to use a \class{VBox} to create a stack of two buttons inside the
window, and one way to use the \var{sender} parameter to influence the
action's outcome according to data obtained from the triggered control.

\verbatiminput{multiplexedaction.py}

\class{VBox}es and \class{HBox}es are used in \Cimarron{} to lay out
widgets. \class{VBox} stacks the widgets inside it on top of each other
from top to bottom, \class{HBox} makes a horizontal row with them, from
left to right. You can have a \class{VBox} inside a \class{HBox} inside
a \class{VBox}, to as many levels as you want. You could have a
\class{VBox} inside a \class{VBox}, but of course it would be rather
pointless.

\section{A four-function simple calculator}

Here's an attempt at constructing a somewhat useful program using what
we know so far about \Cimarron:

\verbatiminput{calculator.py}

As usual, we have a controller for the window, whose constructor method
builds the window's contents from the outside in. We use a label to
display the calculator's input and result, because we want the user to
input data only thorugh the numeric keypad directly below it, and not
using the keyboard. The keypad itself is made up of four columns of
buttons, which we set up using data from a list of dictionaries, one for
each key. Note that in the operation keys we use the fact that the
initializer will accept undefined attributes to associate an operation
to each button. This association is later used by the \method{operate()}
method. The actual functionality of the calculator is then implemented
in three simple methods that modify the calculator's state based on the
user's actions.

The code above has a bug: it allows the user to input a number with more
than one decimal point. Try to find a few different ways of getting rid
of that bug.

\section{Interacting with a model}

\Cimarron{} is meant to help implement applications organized around the
Model-View-Controller pattern. The previous examples have shown
interaction between a controller (\class{MainWindowController}), whose
job is to set up a view (the window and its widgets), complete with the
actions that must occur when something is triggered. The widgets also
know how to interact with a model, over a simple API, to keep track of
the values of particular attributes.

  \subsection{A toy model}

  Since allowing for user interaction to edit a network of objects is
  what \Cimarron{} is all about, here's a toy model for us to play
  with. From this section on, we will be using the following data model:

  \verbatiminput{model.py}

  As you can see, the classes that make up the model are pretty
  simple. All that \Cimarron{} demands of them is that they implement
  the \class{IModel} interface. The \class{Model} class can be derived
  to create classes that automatically implement the Interface.

  There is one-to-many relationship between instances of \class{Person}
  and \class{Address}. This is represented by \class{Person} having a
  list-like (in this case, a true \class{List}) attribute where it can
  store the related instances.

  The last part of the declaration just fills creates a few instances of
  the classes and stores in \class{Person}'s \member{__values__}
  attribute, just so our programs will have something to play with. In a
  real-world program, there would be a mechanism for retrieving and
  storing instances of these classes in a database.

  \subsection{Wiring widgets to the model}

  Here's how we can write a program that displays and edits a
  \class{Person} instance. This program displays a lot of the main
  characteristics that a real \Cimarron{} program would have: user
  feedback and input aids, validation, and editing of live objects.

  \verbatiminput{simplepersoneditor.py}

  As you can see, again the bulk of the code is concerned with creating
  and packing the widgets. This program shows off a few useful
  techniques for programming \Cimarron{} applications. Before you try to
  understand the program fully, you may want to try using it a bit, to
  get a feeling for what it does. Try to answer these questions: what is
  the difference between leaving an \class{Entry} with \kbd{Tab} or
  \kbd{Enter}? Can you set the value of one of the \class{Entry}s to the
  empty string? What happens to the case of letters you enter in the
  \class{Entry}s?

  \subsection{The \member{target} interface}

  From the earlier examples, you may have expected this program to use
  its \member{action} attribute to update the model. However, if you
  check the code, you will probably notice that none of it is concerned
  with moving the values from and to the widgets. The \class{Entry}
  widgets' associated action is \method{checkValues()}, which just reads
  values from the model and formats them for display in a
  \class{Label}.

  As a matter of fact, the widgets don't even need to have any action at
  all in order to be able to interact with the model, and they keep
  synced with it even when the target is not invoked, such as when you
  leave an \class{Entry} with \kbd{Tab}. All the controller has to do is
  set the \class{Widget}'s \member{attribute} attribute to the name of
  one of the model's attributes. Later, when a \member{target} is
  associated with the \class{Widget}, it will automatically display the
  value of the target's attribute, and it will try to edit it according
  to the user's actions.

  Once the window is all set up, it is \class{PersonEditor}'s
  \method{newTarget()} method's responsibility to inform each widget
  which concrete instance they are to interact with at any given
  moment. To that end, \class(PersonEditor) instances keep a list of the
  widgets that it needs to update when its target changes. Once a
  widget's target is set, it's takes responsibility for keeping its
  value in sync with the model.

  \subsection{The \member{target} is independent of the action}

  As shown in the code, a widget can have both an action and an
  attribute associated with it: the action establishes a communications
  channel back to the controller, while the attribute acts as a forward
  communication with the model. For \class{Entry} instances, the action
  is invoked only if the user presses \kbd{Enter}, but the value in the
  model is always updated.

  The \method{checkValues()} method, invoked every time you press the
  ``Check'' button or trigger the action of an \class{Entry} by pressing
  \kbd{Enter}, is meant to show that the actual \class{Person} object's
  values are being modified through the widgets. The program would still
  work perfectly if the \class{Entry}s had no associated action.

  \subsection{Delegation}

  Many \Cimarron{} widgets allow for customization of their behavior
  through a mechanism called \emph{delegation}. Such widgets have a list
  of delegates which can be set at instantiation (such as above), and
  added to afterwards. A delegate is just any object that can answer to
  one or more of the widget's \emph{delegate messages}, which typically
  have names that begin with \member{will_} or \member{did_}. When
  an event that may require cooperation from delegates takes place in
  the widget, it scans its delegate list for objects that can answer to
  the relevant message, and calls them in order. In the code above, we use the
  \member{will_focus_out} delegate message of \class{Entry}
  widgets to enforce proper capitalization of names, get rid of spurious
  whitespace before or after the name, and to keep the user from
  entering empty names.

  Delegate messages whose name begins with \member{will_} are often used
  to influence the way the widget acts, and can be understood as the
  widget asking for permission to proceed. Whenever the \class{Entry} is
  about to lose the input focus for any reason, it scans the delegate
  list for objects that can answer to this message. The object can
  answer with any of the values \member{ForcedNo}, \member{No},
  \member{Unknown}, \member{Yes} or \member{ForcedYes}. The
  \class{Entry} will only proceed with the focus out operation if the
  combined answers of the delegates add up to allowing it. Returning
  either \member{ForcedYes} or \member{ForcedNo} will cause the widget
  to stop traversing the delegate list, and either proceed or abort the
  operation. From the point of view of deciding whether to proceed or
  not with the operation, returning \member{Unknown} is the same as not
  answering to the message. The rules for combining the other values are
  as follows:
  \begin{itemize}
   \item All \member{Unknown}s: proceed
   \item At least one \member{No}, all others \member{Unknow}: abort
   \item At least one \member{Yes}, all others \member{No} or
	 \member{Unknown}: proceed
  \end{itemize}

  Delegate messages whose names begin with \member{did_} are scanned for
  \emph{after} the action has taken place, and are typically used for
  cleanup chores, or to keep other widgets in sync with the one that
  triggered the delegation message.

  Different widgets define different delegation messages, so you must
  check the widget's documentation to find out which delegation messages
  are supported by a widget, and what its parameters are. Since one
  single object can act as the delegate of several widgets (as in the
  code above), the first parameter of the delegate method is always the
  widget.

  \subsection{Layout tuning}

  This program sets the \var{expand} parameter of some of the widgets to
  \class{False} to optimize the window's layout. This parameter, which
  defaults to \class{True}, controls whether the widget will expand in
  the direction of the enclosing box. When it is set to \class{False},
  the widget will only grow to its natural size, eventually allowing
  other widgets to take better advantage of the available window real
  estate. This can be easily seen in action when you resize the window
  using the mouse. You may want to experiment changing its value for
  different widgets, to see what happens.

  \subsection{Multi-valued attributes}

  The list of addresses in the example model requires a way to handle
  attributes that have several values. The \class{Grid} is the right
  tool for this case. A \class{Grid}'s \member{attribute} must be a
  list-like object, which the \class{Grid} will display one element per
  row. Quite often, the elements themselves will be composite values,
  such as in this case. To display the values of each row, the
  \class{Grid} is divided in \class{Columns}. Each column has a
  \member{name}, which is displayed at its top, and an
  \member{attribute}, which specifies the element's attribute that
  can be displayed/edited in the column. The \class{Grid} provides user
  interface mechanisms to add, delete and edit values. Expanding the
  previous example to handle the list of addresses is rather
  straightforward:

  \verbatiminput{personaddresseditor.py}

  The \class{Grid} class supports delegation methods to give controllers
  the ability to fine-tune its behavior with regard to additions,
  deletions and changes.
  
\end{document}
